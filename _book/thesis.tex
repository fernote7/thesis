% This is the Reed College LaTeX thesis template. Most of the work
% for the document class was done by Sam Noble (SN), as well as this
% template. Later comments etc. by Ben Salzberg (BTS). Additional
% restructuring and APA support by Jess Youngberg (JY).
% Your comments and suggestions are more than welcome; please email
% them to cus@reed.edu
%
% See http://web.reed.edu/cis/help/latex.html for help. There are a
% great bunch of help pages there, with notes on
% getting started, bibtex, etc. Go there and read it if you're not
% already familiar with LaTeX.
%
% Any line that starts with a percent symbol is a comment.
% They won't show up in the document, and are useful for notes
% to yourself and explaining commands.
% Commenting also removes a line from the document;
% very handy for troubleshooting problems. -BTS

% As far as I know, this follows the requirements laid out in
% the 2002-2003 Senior Handbook. Ask a librarian to check the
% document before binding. -SN

%%
%% Preamble
%%
% \documentclass{<something>} must begin each LaTeX document
\documentclass[12pt,twoside]{reedthesis}
% Packages are extensions to the basic LaTeX functions. Whatever you
% want to typeset, there is probably a package out there for it.
% Chemistry (chemtex), screenplays, you name it.
% Check out CTAN to see: http://www.ctan.org/
%%
\usepackage{graphicx,latexsym}
\usepackage{amsmath}
\usepackage{amssymb,amsthm}
\usepackage{longtable,booktabs,setspace}
\usepackage{chemarr} %% Useful for one reaction arrow, useless if you're not a chem major
\usepackage[hyphens]{url}
% Added by CII
\usepackage[hidelinks]{hyperref}
\usepackage{lmodern}
\usepackage{float}
\floatplacement{figure}{H}
% End of CII addition
\usepackage{rotating}

% Next line commented out by CII
%%% \usepackage{natbib}
% Comment out the natbib line above and uncomment the following two lines to use the new
% biblatex-chicago style, for Chicago A. Also make some changes at the end where the
% bibliography is included.
%\usepackage{biblatex-chicago}
%\bibliography{thesis}


% Added by CII (Thanks, Hadley!)
% Use ref for internal links
\renewcommand{\hyperref}[2][???]{\autoref{#1}}
\def\chapterautorefname{Chapter}
\def\sectionautorefname{Section}
\def\subsectionautorefname{Subsection}
% End of CII addition

% Added by CII
\usepackage{caption}
\captionsetup{width=5in}
% End of CII addition

% \usepackage{times} % other fonts are available like times, bookman, charter, palatino


% To pass between YAML and LaTeX the dollar signs are added by CII
\title{\textbf{\Huge{Numerical methods for the \\[20pt] Heston model}}}
\author{Fernando O. Teixeira}
% The month and year that you submit your FINAL draft TO THE LIBRARY (May or December)
\date{September 10, 2017}
\division{Applied Mathematics}
\advisor{Hugo Alexander de la Cruz Cancino}
%If you have two advisors for some reason, you can use the following
% Uncommented out by CII
% End of CII addition

%%% Remember to use the correct department!
\department{Mathematics}
% if you're writing a thesis in an interdisciplinary major,
% uncomment the line below and change the text as appropriate.
% check the Senior Handbook if unsure.
%\thedivisionof{The Established Interdisciplinary Committee for}
% if you want the approval page to say "Approved for the Committee",
% uncomment the next line
%\approvedforthe{Committee}

% Added by CII
%%% Copied from knitr
%% maxwidth is the original width if it's less than linewidth
%% otherwise use linewidth (to make sure the graphics do not exceed the margin)
\makeatletter
\def\maxwidth{ %
  \ifdim\Gin@nat@width>\linewidth
    \linewidth
  \else
    \Gin@nat@width
  \fi
}
\makeatother

\renewcommand{\contentsname}{Table of Contents}
% End of CII addition

\setlength{\parskip}{0pt}

% Added by CII

\providecommand{\tightlist}{%
  \setlength{\itemsep}{0pt}\setlength{\parskip}{0pt}}

\Acknowledgements{
Any one who considers arithmetical methods of producing random digits
is, of course, in a state of sin. - John von Neumann
\textbf{\\ \\ \\ \\ \\ \\ \\ \\ \\ \\ \\ \\ \\ \\ \\ \\ \\ \\ \\ \\ \\ \\ \\ \\ \\ \\ \\ \\ \\ \\ \\ \\ \\ \\ \\ \\ \\ \\ \\ \\ \\ \\ \\ \\ \\ \\ \\ \\ \\ \\ \\ \\ \\ \\ \\ \\ \\ \\ \\ \\ \\ \\ \\ }
You get pseudo-order when you seek order; you only get a measure of
order and control when you embrace randomness. --- Nassim Nicholas Taleb
}

\Dedication{
You can have a dedication here if you wish.
}

\Preface{

}

\Abstract{
The preface pretty much says it all. \par  Second paragraph of abstract
starts here.
}

	\usepackage{mathtools}
	\usepackage{cancel}
	\graphicspath{ {figure/} }
	\usepackage{enumitem}
% End of CII addition
%%
%% End Preamble
%%
%
\begin{document}

% Everything below added by CII
      \maketitle
  
  \frontmatter % this stuff will be roman-numbered
  \pagestyle{empty} % this removes page numbers from the frontmatter
      \begin{acknowledgements}
      Any one who considers arithmetical methods of producing random digits
      is, of course, in a state of sin. - John von Neumann
      \textbf{\\ \\ \\ \\ \\ \\ \\ \\ \\ \\ \\ \\ \\ \\ \\ \\ \\ \\ \\ \\ \\ \\ \\ \\ \\ \\ \\ \\ \\ \\ \\ \\ \\ \\ \\ \\ \\ \\ \\ \\ \\ \\ \\ \\ \\ \\ \\ \\ \\ \\ \\ \\ \\ \\ \\ \\ \\ \\ \\ \\ \\ \\ \\ }
      You get pseudo-order when you seek order; you only get a measure of
      order and control when you embrace randomness. --- Nassim Nicholas Taleb
    \end{acknowledgements}
  
      \hypersetup{linkcolor=black}
    \setcounter{tocdepth}{2}
    \tableofcontents
  
      \listoftables
  
      \listoffigures
      \begin{abstract}
      The preface pretty much says it all. \par  Second paragraph of abstract
      starts here.
    \end{abstract}
      \begin{dedication}
      You can have a dedication here if you wish.
    \end{dedication}
  \mainmatter % here the regular arabic numbering starts
  \pagestyle{fancyplain} % turns page numbering back on

  \chapter{\texorpdfstring{altadvisor: `Your Other
  Advisor'}{altadvisor: Your Other Advisor}}\label{altadvisor-your-other-advisor}
  
  \chapter{Literature Review}\label{lt-review}
  
  \chapter{The Heston Model
  Implementation}\label{the-heston-model-implementation}
  
  \chapter{Results}\label{results}
  
  We present here the results of all the implementations that were
  disclosed in the previous section. We perform numerical comparisons
  between all the methods, setting out differences accross number of
  simulations and timesteps.
  
  Heston {[}1{]} gives a closed form used for comparison as the `true'
  option value and enabling the results to be exposed in terms of
  bias\footnote{\(\mathbb{E} \left[ \hat{\alpha} - \alpha \right]\)} and
  RMSE (root mean square error).\footnote{Defined as
    \(\sqrt{\mathbb{E}((\hat{\theta}-\theta)^2)}\)}
  
  The simulaton experiments were performed on a notebook with an Intel(R)
  Core(TM) i7-4500U CPU @ 1.80GHz processor and 8GB of RAM running on a
  linux x86\_64 based OS, Fedora 25. Codes were all written in R 3.4.1
  ``Single Candle'' {[}2{]}.
  
  \clearpage
  \begin{tabular}{l|r}
  \hline
  Variables & Values\\
  \hline
  dt & 0.05\\
  \hline
  k & 6.21\\
  \hline
  r & 0.03\\
  \hline
  rho & -0.70\\
  \hline
  S & 100.00\\
  \hline
  sigma & 0.61\\
  \hline
  t & 0.00\\
  \hline
  tau & 1.00\\
  \hline
  theta & 0.00\\
  \hline
  v & 0.01\\
  \hline
  X & 100.00\\
  \hline
  \end{tabular}
  \chapter{Conclusion}\label{conclusion}
  
  \chapter{Black-Scholes formula}\label{bsformula}
  
  \chapter*{References}\label{references}
  \addcontentsline{toc}{chapter}{References}
  
  \hypertarget{refs}{}
  \hypertarget{ref-heston1993closed}{}
  {[}1{]} S.L. Heston, A closed-form solution for options with stochastic
  volatility with applications to bond and currency options, Review of
  Financial Studies. 6 (1993) 327--343.
  
  \hypertarget{ref-rlang}{}
  {[}2{]} R Core Team, R: A language and environment for statistical
  computing, R Foundation for Statistical Computing, Vienna, Austria,
  2017.


  % Index?

\end{document}
