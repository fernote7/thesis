% This is the Reed College LaTeX thesis template. Most of the work
% for the document class was done by Sam Noble (SN), as well as this
% template. Later comments etc. by Ben Salzberg (BTS). Additional
% restructuring and APA support by Jess Youngberg (JY).
% Your comments and suggestions are more than welcome; please email
% them to cus@reed.edu
%
% See http://web.reed.edu/cis/help/latex.html for help. There are a
% great bunch of help pages there, with notes on
% getting started, bibtex, etc. Go there and read it if you're not
% already familiar with LaTeX.
%
% Any line that starts with a percent symbol is a comment.
% They won't show up in the document, and are useful for notes
% to yourself and explaining commands.
% Commenting also removes a line from the document;
% very handy for troubleshooting problems. -BTS

% As far as I know, this follows the requirements laid out in
% the 2002-2003 Senior Handbook. Ask a librarian to check the
% document before binding. -SN

%%
%% Preamble
%%
% \documentclass{<something>} must begin each LaTeX document
\documentclass[12pt,twoside]{reedthesis}
% Packages are extensions to the basic LaTeX functions. Whatever you
% want to typeset, there is probably a package out there for it.
% Chemistry (chemtex), screenplays, you name it.
% Check out CTAN to see: http://www.ctan.org/
%%
\usepackage{graphicx,latexsym}
\usepackage{amsmath}
\usepackage{amssymb,amsthm}
\usepackage{longtable,booktabs,setspace}
\usepackage{chemarr} %% Useful for one reaction arrow, useless if you're not a chem major
\usepackage[hyphens]{url}
% Added by CII
\usepackage[hidelinks]{hyperref}
\usepackage{lmodern}
\usepackage{float}
\floatplacement{figure}{H}
% End of CII addition
\usepackage{rotating}

% Next line commented out by CII
%%% \usepackage{natbib}
% Comment out the natbib line above and uncomment the following two lines to use the new
% biblatex-chicago style, for Chicago A. Also make some changes at the end where the
% bibliography is included.
%\usepackage{biblatex-chicago}
%\bibliography{thesis}


% Added by CII (Thanks, Hadley!)
% Use ref for internal links
\renewcommand{\hyperref}[2][???]{\autoref{#1}}
\def\chapterautorefname{Chapter}
\def\sectionautorefname{Section}
\def\subsectionautorefname{Subsection}
% End of CII addition

% Added by CII
\usepackage{caption}
\captionsetup{width=5in}
% End of CII addition

% \usepackage{times} % other fonts are available like times, bookman, charter, palatino


% To pass between YAML and LaTeX the dollar signs are added by CII
\title{\textbf{\Huge{Numerical methods for the \\[20pt] Heston model}}}
\author{Fernando O. Teixeira}
% The month and year that you submit your FINAL draft TO THE LIBRARY (May or December)
\date{September 13, 2017}
\division{Applied Mathematics}
\advisor{Hugo Alexander de la Cruz Cancino}
%If you have two advisors for some reason, you can use the following
% Uncommented out by CII
% End of CII addition

%%% Remember to use the correct department!
\department{Mathematics}
% if you're writing a thesis in an interdisciplinary major,
% uncomment the line below and change the text as appropriate.
% check the Senior Handbook if unsure.
%\thedivisionof{The Established Interdisciplinary Committee for}
% if you want the approval page to say "Approved for the Committee",
% uncomment the next line
%\approvedforthe{Committee}

% Added by CII
%%% Copied from knitr
%% maxwidth is the original width if it's less than linewidth
%% otherwise use linewidth (to make sure the graphics do not exceed the margin)
\makeatletter
\def\maxwidth{ %
  \ifdim\Gin@nat@width>\linewidth
    \linewidth
  \else
    \Gin@nat@width
  \fi
}
\makeatother

\renewcommand{\contentsname}{Table of Contents}
% End of CII addition

\setlength{\parskip}{0pt}

% Added by CII

\providecommand{\tightlist}{%
  \setlength{\itemsep}{0pt}\setlength{\parskip}{0pt}}

\Acknowledgements{
Any one who considers arithmetical methods of producing random digits
is, of course, in a state of sin. - John von Neumann
\textbf{\\ \\ \\ \\ \\ \\ \\ \\ \\ \\ \\ \\ \\ \\ \\ \\ \\ \\ \\ \\ \\ \\ \\ \\ \\ \\ \\ \\ \\ \\ \\ \\ \\ \\ \\ \\ \\ \\ \\ \\ \\ \\ \\ \\ \\ \\ \\ \\ \\ \\ \\ \\ \\ \\ \\ \\ \\ \\ \\ \\ \\ \\ \\ }
You get pseudo-order when you seek order; you only get a measure of
order and control when you embrace randomness. --- Nassim Nicholas Taleb
}

\Dedication{
You can have a dedication here if you wish.
}

\Preface{

}

\Abstract{
The preface pretty much says it all. \par  Second paragraph of abstract
starts here.
}

	\usepackage{mathtools}
	\usepackage{cancel}
	\graphicspath{ {figure/} }
	\usepackage{enumitem}
% End of CII addition
%%
%% End Preamble
%%
%
\begin{document}

% Everything below added by CII
      \maketitle
  
  \frontmatter % this stuff will be roman-numbered
  \pagestyle{empty} % this removes page numbers from the frontmatter
      \begin{acknowledgements}
      Any one who considers arithmetical methods of producing random digits
      is, of course, in a state of sin. - John von Neumann
      \textbf{\\ \\ \\ \\ \\ \\ \\ \\ \\ \\ \\ \\ \\ \\ \\ \\ \\ \\ \\ \\ \\ \\ \\ \\ \\ \\ \\ \\ \\ \\ \\ \\ \\ \\ \\ \\ \\ \\ \\ \\ \\ \\ \\ \\ \\ \\ \\ \\ \\ \\ \\ \\ \\ \\ \\ \\ \\ \\ \\ \\ \\ \\ \\ }
      You get pseudo-order when you seek order; you only get a measure of
      order and control when you embrace randomness. --- Nassim Nicholas Taleb
    \end{acknowledgements}
  
      \hypersetup{linkcolor=black}
    \setcounter{tocdepth}{2}
    \tableofcontents
  
      \listoftables
  
      \listoffigures
      \begin{abstract}
      The preface pretty much says it all. \par  Second paragraph of abstract
      starts here.
    \end{abstract}
      \begin{dedication}
      You can have a dedication here if you wish.
    \end{dedication}
  \mainmatter % here the regular arabic numbering starts
  \pagestyle{fancyplain} % turns page numbering back on

  \chapter{\texorpdfstring{altadvisor: `Your Other
  Advisor'}{altadvisor: Your Other Advisor}}\label{altadvisor-your-other-advisor}
  
  \chapter{Literature Review}\label{lt-review}
  
  \chapter{The Heston Model
  Implementation}\label{the-heston-model-implementation}
  
  In section \ref{hes1} we presented Heston's SDE system in one of its
  structures. Another common way {[}1,3,10{]} to write down the system is
  using the property presented in subsection \ref{corr} as in equation
  \eqref{eq:heston2}.
  \begin{align}
  \label{eq:heston2}
  \begin{split}
  dS_t &= \mu S_t dt + \rho \sqrt{V_t} dB_t + \sqrt{1 - \rho^2} \sqrt{V_t} S_t dW_t \\
  dV_t &= k(\theta - V_t)dt + \sigma \sqrt{V_t} dB_t 
  \end{split}
  \end{align}
  \section{Characteristic Function}\label{characteristic-function}
  
  The Heston model characteristic function is firstly presented in the
  1993 Steven Heston's paper {[}8{]} and is described below {[}5{]}:
  \begin{align}
  f(S_t, V_t, t) = e^{A(T-t)+B(T-t)S_t + C(T-t)V_t + i \phi S_t}
  \end{align}
  If we let \(\tau = T-t\), then the explicit form of the Heston
  characteristic function is:
  \begin{align*}
  f(i \phi) &= e^{A(\tau)+B(\tau)S_t + C(\tau)V_t + i \phi S_t} \\
  A(\tau) &= r i \phi \tau + \frac{\kappa \theta}{\sigma^2} \left[ - (\rho \sigma i \phi - \kappa - M) \tau - 2 \ln\left(\frac{1-N e^{M \tau}}{1-N}\right) \right] \\
  B(\tau) &= 0 \\
  C(\tau) &= \frac{(e^{M \tau}-1)(\rho \sigma i \phi - \kappa - M)}{\sigma^2 (1-N e^{M \tau})} \\
  \text{Where:} & \\
  M &= \sqrt{(\rho \sigma i \phi - \kappa)^2 + \sigma^2 (i \phi + \phi^2)} \\
  N &= \frac{\rho \sigma i \phi - \kappa - M}{\rho \sigma i \phi - \kappa + M} \\
  \end{align*}
  This function is the driving force behind the following formula, that
  calculates the fair valur of a European call option at time \(t\), given
  a strike price \(K\), that expires at time \(T\) {[}5{]}:
  \begin{align} 
  \label{eq:cfheston}
  \begin{split}
  C = & \frac{1}{2} S(t) + \frac{e^{-r(T-t)}}{\pi}\int_{0}^{\infty}{\Re \left[ \frac{K^{-i \phi} f(i \phi + 1)}{i \phi} \right] d\phi} \\
  & -Ke^{-r(T-t)}\left( \frac{1}{2} + \frac{1}{\pi} \int_{0}^{\infty}{\Re \left[ \frac{K^{-i \phi} f(i \phi)}{i \phi} \right]}  d\phi \right)
  \end{split}
  \end{align}
  \section{Euler Scheme}\label{euler-scheme}
  
  Given the fact that the underlying asset is temporal dependent upon the
  solution of the SDE's volatility, we simulate the volatility's path
  before the asset's. If the Black-Scholes model enabled using Ito's Lemma
  directly for solving \(S_t\), this equation system requires numerical
  methods. We present here the Euler Scheme - Full Truncation algorithm
  (and compare to other similar schemes) {[}3{]} along with some insights
  on how it was implemented in R. The Euler discretization brings
  approximation paths to stock prices and variance processes. If we set
  \(t_0 = 0 < t_1 < \dots < t_M = T\) as partitions of a time interval of
  \(M\) equal segments of lenght \(\delta t\), we have the following
  discretization for the stock price:
  \begin{align}
  S_{t+1} = S_t + rS_t + \sqrt{V_t} S_t Z_s
  \end{align}
  \noindent
  And for the variance process:
  \begin{align}
  V_{t+1} = f_1(V_{t}) + \kappa (\theta - f_2(V_{t})) + \sigma \sqrt{f_3(V_{t})} Z_v 
  \end{align}
  \noindent
  \(Z_s\) being a standard normal random variable, i.e. \(N\sim(0,1)\), we
  set \(Z_t\) and \(Z_v\) as two independent standard normal random
  variables and \(Z_s\) and \(Z_v\) having correlation \(\rho\). This
  means we can write \(Z_s = \rho Z_v + \sqrt{1-\rho^2} Z_t\).
  
  The immediate observable problem in the proposed discretization scheme
  is that \(V\) can become negative with non-zero probability making the
  computation of \(\sqrt{V_t}\) impossible {[}1{]}. There are several
  proposed fixes that can be used as you can see below:
  \begin{longtable}[t]{lrrr}
  \caption{\label{tab:fullt}Truncation schemes}\\
  \toprule
  Scheme & $f_1(V_{t})$ & $f_2(V_{t})$ & $f_3(V_{t})$\\
  \midrule
  Reflection & $\mid V \mid$ & $\mid V \mid$ & $\mid V \mid$\\
  Partial Truncation & $V$ & $V$ & $V^+$\\
  Full Truncation & $V$ & $V^+$ & $V^+$\\
  \bottomrule
  \end{longtable}
  Where \(V^+ = \max(V,0)\) and \(\mid V \mid\) is the absolute value of
  \(V\).
  
  We chose to fix our discretization using the Full-Truncation (FT) scheme
  and thus, rewrite the equations as follows:
  \begin{align}
  S_{t+1} &= S_t + rS_t + \sqrt{V_{t}^{+}} S_t Z_s \\
  V_{t+1} &= V_t + \kappa (\theta - V_{t}^{+}) + \sigma \sqrt{V_{t}^{+}} Z_v
  \end{align}
  \section{Kahl-Jackel}\label{kahl-jackel}
  
  Kahl-Jackel propose a discretization method they refer to as the ``IJK''
  method {[}1,10{]} that coupled with the implicit Milstein scheme for the
  variance lands the system of equations \eqref{eq:kj1} and \eqref{eq:kj2}. It
  is possible to verify that this discretization always results in
  positive paths for \(V\) if \(4 \kappa \theta > \sigma^2\).
  Unfortunately, this inequality is rarely satisfied when we plug real
  market data to calibrate the parameters.
  \begin{small}
  \begin{align}
  \label{eq:kj1}
  \ln \hat{S}(t + \Delta) &= \ln \hat{S}(t) - \frac{\Delta}{4}\left( \hat{V}(t+\Delta) + \hat{V}(t) \right) + \rho \sqrt{\hat{V}(t)}Z_v\sqrt{\Delta} \\ \nonumber
  &+ \frac{1}{2} \left( \sqrt{\hat{V}(t+\Delta)} + \sqrt{\hat{V}(t)} \right) \left( Z_S \sqrt{\Delta} - \rho Z_V \sqrt{\Delta}\right) + \frac{1}{4} \sigma \rho \Delta \left( Z_{V}^{2} - 1 \right) \\
  \label{eq:kj2}
  \hat{V}(t+\Delta) &= \frac{\hat{V}(t) + \kappa \theta \Delta + \sigma \sqrt{\hat{V}(t)}Z_V \sqrt{\Delta}+ \frac{1}{4}\sigma^2 \Delta \left(Z_V^2-1 \right)}{1+ \kappa \Delta}
  \end{align}
  \end{small}
  \section{Exact Algorithm}\label{exact-algorithm}
  
  In 2006, Broadie-Kaya {[}3{]} propose a method that has a faster
  convergence rate, \(\mathcal{O} \left( s^{-1/2} \right)\) than some of
  the more famous schemes, such as Euler's and Milstein's discretizations.
  They build their idea to generate an exact sample from the distribution
  of the terminal stock price based on numerous papers {[}8{]}. The stock
  price and variance are as follows:
  \begin{align} \label{eq:ea1}
  S_t = S_0 \, exp \left[ \mu t - \frac{1}{2} \int_{0}^{t}{V_s ds} + \rho  \int_{0}^{t}{\sqrt{V_s d B_s}} + \sqrt{1 - \rho^2} \int_{0}^{t}{\sqrt{V_s} dW_s}\right]
  \end{align}
  The squared volatility of the variance process is:
  \begin{align} \label{eq:ea2}
  V_t = V_0 + \kappa \theta t - \kappa \int_{0}^{t}{V_s ds} + \sigma \int_{0}^{t}{\sqrt{V_s dB_s}}
  \end{align}
  The algorithm used to generate the model consists in four steps as
  follows:
  \begin{itemize}
  \item [\textit{Step} 1.] Generate a sample of $V_t$ given $V_0$
  \item [\textit{Step} 2.] Generate a sample of $\displaystyle \int_0^t V_sds$ given $V_t$, $V_0$
  \item [\textit{Step} 3.] Compute $displaystyle \int_0^t \sqrt{V_s}dB_s$ given $V_t$, $V_0$ and $\int_0^t V_sds$
  \item [\textit{Step} 4.] Generate a sample from the probability distribution of $S_t$, given $displaystyle \int_0^t \sqrt{V_s}dB_s$ and $displaystyle \int_0^t V_sds$
  \end{itemize}
  \subsection{\texorpdfstring{Generate a sample of \(V_t\) given
  \(V_0\)}{Generate a sample of V\_t given V\_0}}\label{generate-a-sample-of-v_t-given-v_0}
  
  The distribution of \(V_t\) given \(V_0\) for \(0 < t\) is a noncentral
  chi-squared distribution {[}2,4{]}:
  
  \[V_t = \frac{\sigma^2 (1-e^{- \kappa t})}{4 \kappa} \mathcal{X}_{\delta}^{2} \left( \frac{4 \kappa e^{- \kappa t}}{\sigma^2 (1- e^{- \kappa t})} \times V_0\right)\]
  
  where \(\delta = \frac{4 \theta \kappa}{\sigma^2}\) and
  \(\mathcal{X}_{\delta}^{2}(\lambda)\) denotes a noncentral chi-squared
  random variable with \(\delta\) degrees of freedom and \(\lambda\) as
  its noncentrality parameter.
  
  Broadie and Kaya {[}3{]} sample generating Poisson and gamma
  distributions as in Johnson et al. {[}9{]}. We used the built-in
  function in R {[}11{]} which uses this exact method for sampling.
  
  \subsection{\texorpdfstring{Generate a sample of \(\int_0^t V_sds\)
  given \(V_t\),
  \(V_0\)}{Generate a sample of \textbackslash{}int\_0\^{}t V\_sds given V\_t, V\_0}}\label{generate-a-sample-of-int_0t-v_sds-given-v_t-v_0}
  
  After generating \(V_t\), we follow the instructions in {[}3,9{]}. We
  use the characteristic function \eqref{eq:phi} to compute the probability
  density function \(F(x)\).
  \begin{align} \label{eq:phi}
  \begin{split}
  \Phi(a) &= \mathbb{E}\left[ exp \left( ia \int_{0}^{t}{V_s ds} \mid V_0,V_t \right)  \right] \\[10pt]
  &= \frac{\gamma(a)e^{(-1/2)(\gamma(a)- \kappa) t} (1 - e^{- \kappa t})}{\kappa (1 - e^{- \gamma(a) t})} \\[10pt]
  &\times exp \left\{\frac{V_0 + V_t}{\sigma^2} \left[ \frac{\kappa (1 + e^{- \kappa t})}{1 - e^{- \kappa t}} - \frac{\gamma(a) (1 + e^{- \gamma(a) t})}{1 - e^{- \gamma(a) t}} \right] \right\} \\[10pt]
  &\times \frac{I_{0.5\delta - 1} \left[ \sqrt{V_0 V_t} \frac{4 \gamma(a) e^{-0.5 \gamma(a) t}}{\sigma^2 (1 - e^{- \gamma(a) t})} \right]}{I_{0.5\delta - 1}  \left[ \sqrt{V_0 V_t} \frac{4 \kappa e^{-0.5 \kappa t}}{\sigma^2 (1 - e^{- \kappa t})} \right]}
  \end{split}
  \end{align}
  where \(\gamma(a) = \sqrt{\kappa^2 - 2 \sigma^2 i a}\), \(\delta\) was
  previously defined and \(I_v(x)\) is the modified Bessel function of the
  first kind.
  
  The probability distribution function is obtained in {[}2,3{]} by
  Fourier inversions using Feller {[}6{]}. We use the approach in
  Gil-Pelaez {[}7{]}, equation \eqref{eq:fourier}. We define \(V(u,t)\) the
  random variable with the same distribution as the integral
  \(\int_{u}^{t}{V_s ds}\), conditional on \(V_u\) and \(V_t\):
  \begin{align} \label{eq:fourier}
  F(x) \equiv Pr \left\{ V(u,t) \leq x \right\} = F_{X}(x)={\frac {1}{2}}-{\frac {1}{\pi }}\int _{0}^{\infty }{\frac {\operatorname {Im} [e^{-iux} phi (u)]}{u}}\,du
  \end{align}
  \(\operatorname {Im}\) denotes the imaginary part of
  \(e^{-iux} phi (u)\). Equation \eqref{eq:fourier} is computed numerically
  and we then sample it by inversion.
  
  Furthermore, we also introduce a simpler version for this step, that
  computes this integral approximation, using the solution
  \(\int_{u}^{t}{V_s ds} = \frac{1}{2} \left( V_u + V_t \right)\)
  
  \subsection{\texorpdfstring{Compute \(\int_0^t \sqrt{V_s}dB_s\) given
  \(V_t\), \(V_0\) and
  \(\int_0^t V_sds\)}{Compute \textbackslash{}int\_0\^{}t \textbackslash{}sqrt\{V\_s\}dB\_s given V\_t, V\_0 and \textbackslash{}int\_0\^{}t V\_sds}}\label{compute-int_0t-sqrtv_sdb_s-given-v_t-v_0-and-int_0t-v_sds}
  
  From equation \eqref{eq:ea2} we are now able to compute this integral.
  \begin{align} \label{eq:ea3}
  \int_{0}^{t}{\sqrt{V_s dB_s}} = \frac{V_t - V_0 - \kappa \theta t + \kappa \int_{0}^{t}{V_s ds}}{\sigma} 
  \end{align}
  The last step of the algorithm consists of computing the conditional
  distribution of \(log S_t\) based on the fact that the process for
  \(V_t\) is independent from \(dB_t\), and the distribution of
  \(\int_0^t{\sqrt{V_s} dB_s}\) is normal with mean \(0\) and variance
  \(\int_0^t{V_s ds}\), given \(V_t\).
  
  \[m(u,t) = \log S_0 + \left[ \mu t - \frac{1}{2} \int_{0}^{t}{V_s ds} + \rho  \int_{0}^{t}{\sqrt{V_s d B_s}} + \sqrt{1 - \rho^2} \int_{0}^{t}{\sqrt{V_s} dW_s}\right]\]
  
  and variance
  
  \[\sigma^2(0,t) = \left( 1 - \rho^2 \right) \int_0^t{V_s ds}\]
  
  We generate the \(S_t\) sample using a standard normal random variable
  \(Z\) and set:
  
  \[S_t = e^{m(0,t) + \sigma (0,t) Z}\]
  
  \subsection{Limitations}\label{limitations}
  
  The biggest limitation this scheme presents is that the second step is
  computationally costly. It demands the inversion of the
  \(\displaystyle \int_0^t V_sds \mid V_t, \, V_0\)
  
  \chapter{This chunk ensures that the thesisdown package
  is}\label{this-chunk-ensures-that-the-thesisdown-package-is}
  
  \chapter{Conclusion}\label{conclusion}
  
  \chapter{Black-Scholes formula}\label{bsformula}
  
  \chapter*{References}\label{references}
  \addcontentsline{toc}{chapter}{References}
  
  \hypertarget{refs}{}
  \hypertarget{ref-andersen}{}
  {[}1{]} L.B. Andersen, Efficient simulation of the heston stochastic
  volatility model, (2007).
  
  \hypertarget{ref-baldeaux}{}
  {[}2{]} J. Baldeaux, E. Platen, Functionals of multidimensional
  diffusions with applications to finance, Springer Science \& Business
  Media, 2013.
  
  \hypertarget{ref-broadie2006exact}{}
  {[}3{]} M. Broadie, Ö. Kaya, Exact simulation of stochastic volatility
  and other affine jump diffusion processes, Operations Research. 54
  (2006) 217--231.
  
  \hypertarget{ref-cox1985theory}{}
  {[}4{]} J.C. Cox, J.E. Ingersoll Jr, S.A. Ross, A theory of the term
  structure of interest rates, Econometrica: Journal of the Econometric
  Society. (1985) 385--407.
  
  \hypertarget{ref-dunn2014estimating}{}
  {[}5{]} R. Dunn, P. Hauser, T. Seibold, H. Gong, Estimating option
  prices with heston's stochastic volatility model, (2014).
  
  \hypertarget{ref-feller1971introduction}{}
  {[}6{]} W. Feller, Introduction to the theory of probability and its
  applications, vol. 2, II (2. Ed.) New York: Wiley. (1971).
  
  \hypertarget{ref-gil1951note}{}
  {[}7{]} J. Gil-Pelaez, Note on the inversion theorem, Biometrika. 38
  (1951) 481--482.
  
  \hypertarget{ref-heston1993closed}{}
  {[}8{]} S.L. Heston, A closed-form solution for options with stochastic
  volatility with applications to bond and currency options, Review of
  Financial Studies. 6 (1993) 327--343.
  
  \hypertarget{ref-johnson1995}{}
  {[}9{]} N.L. Johnson, S. Kotz, N. Balakrishnan, Continuous univariate
  distributions, vol. 2 of wiley series in probability and mathematical
  statistics: Applied probability and statistics, (1995).
  
  \hypertarget{ref-kahl2006fast}{}
  {[}10{]} C. Kahl, P. Jäckel, Fast strong approximation monte carlo
  schemes for stochastic volatility models, Quantitative Finance. 6 (2006)
  513--536.
  
  \hypertarget{ref-rlang}{}
  {[}11{]} R Core Team, R: A language and environment for statistical
  computing, R Foundation for Statistical Computing, Vienna, Austria,
  2017.
  
  \hypertarget{ref-romano1997}{}
  {[}12{]} M. Romano, N. Touzi, Contingent claims and market completeness
  in a stochastic volatility model, Mathematical Finance. 7 (1997)
  399--412.
  
  \hypertarget{ref-scott1996}{}
  {[}13{]} L. Scott, Simulating a multi-factor term structure model over
  relatively long discrete time periods, in: Proceedings of the Iafe First
  Annual Computational Finance Conference, 1996.
  
  \hypertarget{ref-willard1997}{}
  {[}14{]} G.A. Willard, Calculating prices and sensitivities for
  path-independent derivatives securities in multifactor models, The
  Journal of Derivatives. 5 (1997) 45--61.


  % Index?

\end{document}
