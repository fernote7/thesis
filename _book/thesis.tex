% This is the Reed College LaTeX thesis template. Most of the work
% for the document class was done by Sam Noble (SN), as well as this
% template. Later comments etc. by Ben Salzberg (BTS). Additional
% restructuring and APA support by Jess Youngberg (JY).
% Your comments and suggestions are more than welcome; please email
% them to cus@reed.edu
%
% See http://web.reed.edu/cis/help/latex.html for help. There are a
% great bunch of help pages there, with notes on
% getting started, bibtex, etc. Go there and read it if you're not
% already familiar with LaTeX.
%
% Any line that starts with a percent symbol is a comment.
% They won't show up in the document, and are useful for notes
% to yourself and explaining commands.
% Commenting also removes a line from the document;
% very handy for troubleshooting problems. -BTS

% As far as I know, this follows the requirements laid out in
% the 2002-2003 Senior Handbook. Ask a librarian to check the
% document before binding. -SN

%%
%% Preamble
%%
% \documentclass{<something>} must begin each LaTeX document
\documentclass[12pt,twoside]{reedthesis}
% Packages are extensions to the basic LaTeX functions. Whatever you
% want to typeset, there is probably a package out there for it.
% Chemistry (chemtex), screenplays, you name it.
% Check out CTAN to see: http://www.ctan.org/
%%
\usepackage{graphicx,latexsym}
\usepackage{amsmath}
\usepackage{amssymb,amsthm}
\usepackage{longtable,booktabs,setspace}
\usepackage{chemarr} %% Useful for one reaction arrow, useless if you're not a chem major
\usepackage[hyphens]{url}
% Added by CII
\usepackage[hidelinks]{hyperref}
\usepackage{lmodern}
\usepackage{float}
\floatplacement{figure}{H}
% End of CII addition
\usepackage{rotating}

% Next line commented out by CII
%%% \usepackage{natbib}
% Comment out the natbib line above and uncomment the following two lines to use the new
% biblatex-chicago style, for Chicago A. Also make some changes at the end where the
% bibliography is included.
%\usepackage{biblatex-chicago}
%\bibliography{thesis}


% Added by CII (Thanks, Hadley!)
% Use ref for internal links
\renewcommand{\hyperref}[2][???]{\autoref{#1}}
\def\chapterautorefname{Chapter}
\def\sectionautorefname{Section}
\def\subsectionautorefname{Subsection}
% End of CII addition

% Added by CII
\usepackage{caption}
\captionsetup{width=5in}
% End of CII addition

% \usepackage{times} % other fonts are available like times, bookman, charter, palatino


% To pass between YAML and LaTeX the dollar signs are added by CII
\title{\textbf{\Huge{Numerical methods for stochastic volatility models: \\[20pt] Heston model}}}
\author{Fernando O. Teixeira}
% The month and year that you submit your FINAL draft TO THE LIBRARY (May or December)
\date{August 20, 2017}
\division{Applied Mathematics}
\advisor{Hugo Alexander de la Cruz Cancino}
%If you have two advisors for some reason, you can use the following
% Uncommented out by CII
% End of CII addition

%%% Remember to use the correct department!
\department{Mathematics}
% if you're writing a thesis in an interdisciplinary major,
% uncomment the line below and change the text as appropriate.
% check the Senior Handbook if unsure.
%\thedivisionof{The Established Interdisciplinary Committee for}
% if you want the approval page to say "Approved for the Committee",
% uncomment the next line
%\approvedforthe{Committee}

% Added by CII
%%% Copied from knitr
%% maxwidth is the original width if it's less than linewidth
%% otherwise use linewidth (to make sure the graphics do not exceed the margin)
\makeatletter
\def\maxwidth{ %
  \ifdim\Gin@nat@width>\linewidth
    \linewidth
  \else
    \Gin@nat@width
  \fi
}
\makeatother

\renewcommand{\contentsname}{Table of Contents}
% End of CII addition

\setlength{\parskip}{0pt}

% Added by CII

\providecommand{\tightlist}{%
  \setlength{\itemsep}{0pt}\setlength{\parskip}{0pt}}

\Acknowledgements{
Any one who considers arithmetical methods of producing random digits
is, of course, in a state of sin. - John von Neumann
\textbf{\\ \\ \\ \\ \\ \\ \\ \\ \\ \\ \\ \\ \\ \\ \\ \\ \\ \\ \\ \\ \\ \\ \\ \\ \\ \\ \\ \\ \\ \\ \\ \\ \\ \\ \\ \\ \\ \\ \\ \\ \\ \\ \\ \\ \\ \\ \\ \\ \\ \\ \\ \\ \\ \\ \\ \\ \\ \\ \\ \\ \\ \\ \\ }
You get pseudo-order when you seek order; you only get a measure of
order and control when you embrace randomness. --- Nassim Nicholas Taleb
}

\Dedication{
You can have a dedication here if you wish.
}

\Preface{

}

\Abstract{
The preface pretty much says it all. \par  Second paragraph of abstract
starts here.
}

	\usepackage{mathtools}
	\usepackage{cancel}
	\graphicspath{ {figure/} }
	\usepackage{enumitem}
% End of CII addition
%%
%% End Preamble
%%
%
\begin{document}

% Everything below added by CII
      \maketitle
  
  \frontmatter % this stuff will be roman-numbered
  \pagestyle{empty} % this removes page numbers from the frontmatter
      \begin{acknowledgements}
      Any one who considers arithmetical methods of producing random digits
      is, of course, in a state of sin. - John von Neumann
      \textbf{\\ \\ \\ \\ \\ \\ \\ \\ \\ \\ \\ \\ \\ \\ \\ \\ \\ \\ \\ \\ \\ \\ \\ \\ \\ \\ \\ \\ \\ \\ \\ \\ \\ \\ \\ \\ \\ \\ \\ \\ \\ \\ \\ \\ \\ \\ \\ \\ \\ \\ \\ \\ \\ \\ \\ \\ \\ \\ \\ \\ \\ \\ \\ }
      You get pseudo-order when you seek order; you only get a measure of
      order and control when you embrace randomness. --- Nassim Nicholas Taleb
    \end{acknowledgements}
  
      \hypersetup{linkcolor=black}
    \setcounter{tocdepth}{2}
    \tableofcontents
  
      \listoftables
  
      \listoffigures
      \begin{abstract}
      The preface pretty much says it all. \par  Second paragraph of abstract
      starts here.
    \end{abstract}
      \begin{dedication}
      You can have a dedication here if you wish.
    \end{dedication}
  \mainmatter % here the regular arabic numbering starts
  \pagestyle{fancyplain} % turns page numbering back on

  \chapter{\texorpdfstring{altadvisor: `Your Other
  Advisor'}{altadvisor: Your Other Advisor}}\label{altadvisor-your-other-advisor}
  
  \chapter{Literature Review}\label{lt-review}
  
  \chapter{The Heston Model
  Implementation}\label{the-heston-model-implementation}
  
  In section \ref{hes1} we presented Heston's SDE system in one of its
  structures. Another common way {[}1,2,5{]} to write down the system is
  using the property presented in \ref{corr} as in equation
  \eqref{eq:heston2}.
  \begin{align}
  \label{eq:heston2}
  \begin{split}
  dS_t &= \mu S_t dt + \rho \sqrt{V_t} dB_t + \sqrt{1 - \rho^2} \sqrt{V_t} S_t dW_t \\
  dV_t &= k(\theta - V_t)dt + \sigma \sqrt{V_t} dB_t 
  \end{split}
  \end{align}
  \section{Characteristic Function}\label{characteristic-function}
  
  The Heston model characteristic function is firstly presented in the
  1993 Steven Heston's paper {[}4{]} and is described below {[}3{]}:
  \begin{align}
  f(S_t, V_t, t) = e^{A(T-t)+B(T-t)S_t + C(T-t)V_t + i \phi S_t}
  \end{align}
  If we let \(\tau = T-t\), then the explicit form of the Heston
  characteristic function is:
  \begin{align*}
  f(i \phi) &= e^{A(\tau)+B(\tau)S_t + C(\tau)V_t + i \phi S_t} \\
  A(\tau) &= r i \phi \tau + \frac{\kappa \theta}{\sigma^2} \left[ - (\rho \sigma i \phi - \kappa - M) \tau - 2 \ln\left(\frac{1-N e^{M \tau}}{1-N}\right) \right] \\
  B(\tau) &= 0 \\
  C(\tau) &= \frac{(e^{M \tau}-1)(\rho \sigma i \phi - \kappa - M)}{\sigma^2 (1-N e^{M \tau})} \\
  \text{Where:} & \\
  M &= \sqrt{(\rho \sigma i \phi - \kappa)^2 + \sigma^2 (i \phi + \phi^2)} \\
  N &= \frac{\rho \sigma i \phi - \kappa - M}{\rho \sigma i \phi - \kappa + M} \\
  \end{align*}
  This function is the driving force behind the following formula, that
  calculates the fair valur of a European call option at time \(t\), given
  a strike price \(K\), that expires at time \(T\) {[}3{]}:
  \begin{align} 
  \label{eq:cfheston}
  \begin{split}
  C = & \frac{1}{2} S(t) + \frac{e^{-r(T-t)}}{\pi}\int_{0}^{\infty}{\Re \left[ \frac{K^{-i \phi} f(i \phi + 1)}{i \phi} \right] d\phi} \\
  & -Ke^{-r(T-t)}\left( \frac{1}{2} + \frac{1}{\pi} \int_{0}^{\infty}{\Re \left[ \frac{K^{-i \phi} f(i \phi)}{i \phi} \right]}  d\phi \right)
  \end{split}
  \end{align}
  \section{Euler Scheme - Full
  Truncation}\label{euler-scheme---full-truncation}
  
  We present here the Euler Scheme - Full Truncation algorithm {[}2{]}
  along with some insights on how it was implemented in R. The Euler
  discretization brings approximation paths to stock prices and variance
  processes. If we set \(t_0 = 0 < t_1 < \dots < t_M = T\) as partitions
  of a time interval of \(M\) equal segments of lenght \(\delta t\), we
  have the following discretization for the stock price:
  \begin{align}
  S_{t+1} = S_t + rS_t + \sqrt{V_t} S_t Z_s
  \end{align}
  \noindent
  And for the variance process:
  \begin{align}
  V_{t+1} = V_t + \kappa (\theta - V_t) + \sigma \sqrt{V_t} Z_v
  \end{align}
  \noindent
  \(Z_s\) being a standard normal random variable, i.e. \(N\sim(0,1)\), we
  set \(Z_t\) and \(Z_v\) as two independent standard normal random
  variables and \(Z_s\) and \(Z_v\) having correlation \(\rho\). This
  means we can write \(Z_s = \rho Z_v + \sqrt{1-\rho^2} Z_t\)
  
  The immediate observable problem in the proposed discretization scheme
  is that \(V\) can become negative with non-zero probability making the
  computation of \(\sqrt{V_t}\) impossible {[}1{]}. There are several
  proposed fixes that can be used, we chose the Full-Truncation (FT) and
  rewrite the equations as follows:
  \begin{align}
  S_{t+1} &= S_t + rS_t + \sqrt{V_{t}^{+}} S_t Z_s \\
  V_{t+1} &= V_t + \kappa (\theta - V_{t}^{+}) + \sigma \sqrt{V_{t}^{+}} Z_v
  \end{align}
  \noindent
  Where we use the notation \(V_{t}^{+} = \max(V_{t}, 0)\).
  
  \section{Kahl-Jackel}\label{kahl-jackel}
  
  Kahl-Jackel propose a discretization method they refer to as the ``IJK''
  method {[}1,5{]} that coupled with the implicit Milstein scheme for the
  variance lands the system of equations \eqref{eq:kj1} and \eqref{eq:kj2}. It
  is possible to verify that this discretization always results in
  positive paths for \(V\) if \(4 \kappa \theta > \sigma^2\).
  Unfortunately, this inequality is rarely satisfied when we plug real
  market data to calibrate the parameters.
  \begin{small}
  \begin{align}
  \label{eq:kj1}
  \ln \hat{S}(t + \Delta) &= \ln \hat{S}(t) - \frac{\Delta}{4}\left( \hat{V}(t+\Delta) + \hat{V}(t) \right) + \rho \sqrt{\hat{V}(t)}Z_v\sqrt{\Delta} \\ \nonumber
  &+ \frac{1}{2} \left( \sqrt{\hat{V}(t+\Delta)} + \sqrt{\hat{V}(t)} \right) \left( Z_S \sqrt{\Delta} - \rho Z_V \sqrt{\Delta}\right) + \frac{1}{4} \sigma \rho \Delta \left( Z_{V}^{2} - 1 \right) \\
  \label{eq:kj2}
  \hat{V}(t+\Delta) &= \frac{\hat{V}(t) + \kappa \theta \Delta + \sigma \sqrt{\hat{V}(t)}Z_V \sqrt{\Delta}+ \frac{1}{4}\sigma^2 \Delta \left(Z_V^2-1 \right)}{1+ \kappa \Delta}
  \end{align}
  \end{small}
  \section{Exact Algorithm}\label{exact-algorithm}
  
  In 2006, Broadie-Kaya {[}2{]} propose a method that has a faster
  convergence rate, \(\mathcal{O} \left( s^{-1/2} \right)\) than some of
  the more famous schemes, such as Euler's and Milstein's discretizations.
  They build their idea to generate an exact sample from the distribution
  of the terminal stock price based on numerous papers {[}4{]}.
  
  The algorithm used to generate the model consists in four steps as
  follows:
  \begin{itemize}
  \item [\textit{Step} 1.] Generate a sample of $V_t$ given $V_0$
  \item [\textit{Step} 2.] Generate a sample of $\int_0^t V_sds$ given $V_t$
  \item [\textit{Step} 3.] Compute $\int_0^t \sqrt{V_s}dB_s$
  \item [\textit{Step} 4.] teste
  \end{itemize}
  \clearpage
  
  \chapter{This chunk ensures that the thesisdown package
  is}\label{this-chunk-ensures-that-the-thesisdown-package-is}
  
  \chapter{Conclusion}\label{conclusion}
  
  \chapter{Placeholder}\label{placeholder}
  
  \chapter*{References}\label{references}
  \addcontentsline{toc}{chapter}{References}
  
  \hypertarget{refs}{}
  \hypertarget{ref-andersen}{}
  {[}1{]} L.B. Andersen, Efficient simulation of the heston stochastic
  volatility model, (2007).
  
  \hypertarget{ref-broadie2006exact}{}
  {[}2{]} M. Broadie, Ö. Kaya, Exact simulation of stochastic volatility
  and other affine jump diffusion processes, Operations Research. 54
  (2006) 217--231.
  
  \hypertarget{ref-dunn2014estimating}{}
  {[}3{]} R. Dunn, P. Hauser, T. Seibold, H. Gong, Estimating option
  prices with heston's stochastic volatility model, (2014).
  
  \hypertarget{ref-heston1993closed}{}
  {[}4{]} S.L. Heston, A closed-form solution for options with stochastic
  volatility with applications to bond and currency options, Review of
  Financial Studies. 6 (1993) 327--343.
  
  \hypertarget{ref-kahl2006fast}{}
  {[}5{]} C. Kahl, P. Jäckel, Fast strong approximation monte carlo
  schemes for stochastic volatility models, Quantitative Finance. 6 (2006)
  513--536.
  
  \hypertarget{ref-romano1997}{}
  {[}6{]} M. Romano, N. Touzi, Contingent claims and market completeness
  in a stochastic volatility model, Mathematical Finance. 7 (1997)
  399--412.
  
  \hypertarget{ref-scott1996}{}
  {[}7{]} L. Scott, Simulating a multi-factor term structure model over
  relatively long discrete time periods, in: Proceedings of the Iafe First
  Annual Computational Finance Conference, 1996.
  
  \hypertarget{ref-willard1997}{}
  {[}8{]} G.A. Willard, Calculating prices and sensitivities for
  path-independent derivatives securities in multifactor models, The
  Journal of Derivatives. 5 (1997) 45--61.


  % Index?

\end{document}
