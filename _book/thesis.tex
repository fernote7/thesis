% This is the Reed College LaTeX thesis template. Most of the work
% for the document class was done by Sam Noble (SN), as well as this
% template. Later comments etc. by Ben Salzberg (BTS). Additional
% restructuring and APA support by Jess Youngberg (JY).
% Your comments and suggestions are more than welcome; please email
% them to cus@reed.edu
%
% See http://web.reed.edu/cis/help/latex.html for help. There are a
% great bunch of help pages there, with notes on
% getting started, bibtex, etc. Go there and read it if you're not
% already familiar with LaTeX.
%
% Any line that starts with a percent symbol is a comment.
% They won't show up in the document, and are useful for notes
% to yourself and explaining commands.
% Commenting also removes a line from the document;
% very handy for troubleshooting problems. -BTS

% As far as I know, this follows the requirements laid out in
% the 2002-2003 Senior Handbook. Ask a librarian to check the
% document before binding. -SN

%%
%% Preamble
%%
% \documentclass{<something>} must begin each LaTeX document
\documentclass[12pt,twoside]{reedthesis}
% Packages are extensions to the basic LaTeX functions. Whatever you
% want to typeset, there is probably a package out there for it.
% Chemistry (chemtex), screenplays, you name it.
% Check out CTAN to see: http://www.ctan.org/
%%
\usepackage{graphicx,latexsym}
\usepackage{amsmath}
\usepackage{amssymb,amsthm}
\usepackage{longtable,booktabs,setspace}
\usepackage{chemarr} %% Useful for one reaction arrow, useless if you're not a chem major
\usepackage[hyphens]{url}
% Added by CII
\usepackage[hidelinks]{hyperref}
\usepackage{lmodern}
\usepackage{float}
\floatplacement{figure}{H}
% End of CII addition
\usepackage{rotating}

% Next line commented out by CII
%%% \usepackage{natbib}
% Comment out the natbib line above and uncomment the following two lines to use the new
% biblatex-chicago style, for Chicago A. Also make some changes at the end where the
% bibliography is included.
%\usepackage{biblatex-chicago}
%\bibliography{thesis}


% Added by CII (Thanks, Hadley!)
% Use ref for internal links
\renewcommand{\hyperref}[2][???]{\autoref{#1}}
\def\chapterautorefname{Chapter}
\def\sectionautorefname{Section}
\def\subsectionautorefname{Subsection}
% End of CII addition

% Added by CII
\usepackage{caption}
\captionsetup{width=5in}
% End of CII addition

% \usepackage{times} % other fonts are available like times, bookman, charter, palatino


% To pass between YAML and LaTeX the dollar signs are added by CII
\title{\textbf{\Huge{Numerical methods for stochastic volatility models: \\[20pt] Heston model}}}
\author{Fernando O. Teixeira}
% The month and year that you submit your FINAL draft TO THE LIBRARY (May or December)
\date{julho 23, 2017}
\division{Applied Mathematics}
\advisor{Hugo de la Cruz}
%If you have two advisors for some reason, you can use the following
% Uncommented out by CII
% End of CII addition

%%% Remember to use the correct department!
\department{Mathematics}
% if you're writing a thesis in an interdisciplinary major,
% uncomment the line below and change the text as appropriate.
% check the Senior Handbook if unsure.
%\thedivisionof{The Established Interdisciplinary Committee for}
% if you want the approval page to say "Approved for the Committee",
% uncomment the next line
%\approvedforthe{Committee}

% Added by CII
%%% Copied from knitr
%% maxwidth is the original width if it's less than linewidth
%% otherwise use linewidth (to make sure the graphics do not exceed the margin)
\makeatletter
\def\maxwidth{ %
  \ifdim\Gin@nat@width>\linewidth
    \linewidth
  \else
    \Gin@nat@width
  \fi
}
\makeatother

\renewcommand{\contentsname}{Table of Contents}
% End of CII addition

\setlength{\parskip}{0pt}

% Added by CII

\providecommand{\tightlist}{%
  \setlength{\itemsep}{0pt}\setlength{\parskip}{0pt}}

\Acknowledgements{
I want to thank a few people.
}

\Dedication{
You can have a dedication here if you wish.
}

\Preface{

}

\Abstract{
The preface pretty much says it all. \par  Second paragraph of abstract
starts here.
}

	\usepackage{amsthm}
	\usepackage{cancel}
	\graphicspath{ {figure/} }
% End of CII addition
%%
%% End Preamble
%%
%
\begin{document}

% Everything below added by CII
      \maketitle
  
  \frontmatter % this stuff will be roman-numbered
  \pagestyle{empty} % this removes page numbers from the frontmatter
      \begin{acknowledgements}
      I want to thank a few people.
    \end{acknowledgements}
  
      \hypersetup{linkcolor=black}
    \setcounter{tocdepth}{2}
    \tableofcontents
  
      \listoftables
  
      \listoffigures
      \begin{abstract}
      The preface pretty much says it all. \par  Second paragraph of abstract
      starts here.
    \end{abstract}
      \begin{dedication}
      You can have a dedication here if you wish.
    \end{dedication}
  \mainmatter % here the regular arabic numbering starts
  \pagestyle{fancyplain} % turns page numbering back on

  \chapter{\texorpdfstring{altadvisor: `Your Other
  Advisor'}{altadvisor: Your Other Advisor}}\label{altadvisor-your-other-advisor}
  
  \chapter{Literature Review}\label{lt-review}
  
  \chapter{The Heston Model
  Implementation}\label{the-heston-model-implementation}
  
  \section{Characteristic Function}\label{characteristic-function}
  
  The Heston model characteristic function is firstly presented in the
  1993 Steven Heston's paper {[}1{]} and is described below {[}2{]}:
  \begin{align}
  f(S_t, V_t, t) = e^{A(T-t)+B(T-t)S_t + C(T-t)V_t + i \phi S_t}
  \end{align}
  If we let \(\tau = T-t\), then the explicit form of the Heston
  characteristic function is:
  \begin{align*}
  f(i \phi) &= e^{A(\tau)+B(\tau)S_t + C(\tau)V_t + i \phi S_t} \\
  A(\tau) &= r i \phi \tau + \frac{k \theta}{\sigma^2} \left[ - (\rho \sigma i \phi - k - M) \tau - 2 \ln\left(\frac{1-N e^{M \tau}}{1-N}\right) \right] \\
  B(\tau) &= 0 \\
  C(\tau) &= \frac{(e^{M \tau}-1)(\rho \sigma i \phi - k - M)}{\sigma^2 (1-N e^{M \tau})} \\
  \text{Where:} & \\
  M &= \sqrt{(\rho \sigma i \phi - k)^2 + \sigma^2 (i \phi + \phi^2)} \\
  N &= \frac{\rho \sigma i \phi - k - M}{\rho \sigma i \phi - k + M} \\
  \end{align*}
  This function is the driving force behind the following formula, that
  calculates the fair valur of a European call option at time \(t\), given
  a strike price \(K\), that expires at time \(T\) {[}2{]}:
  \begin{align} 
  \label{eq:cfheston}
  \begin{split}
  C = & \frac{1}{2} S(t) + \frac{e^{-r(T-t)}}{\pi}\int_{0}^{\infty}{\Re \left[ \frac{K^{-i \phi} f(i \phi + 1)}{i \phi} \right] d\phi} \\
  & -Ke^{-r(T-t)}\left( \frac{1}{2} + \frac{1}{\pi} \int_{0}^{\infty}{\Re \left[ \frac{K^{-i \phi} f(i \phi)}{i \phi} \right]}  d\phi \right)
  \end{split}
  \end{align}
  \section{Euler Scheme - Full
  Truncation}\label{euler-scheme---full-truncation}
  
  We present here the Euler Scheme - Full Truncation algorithm {[}3{]}
  along with some insights on how it was implemented in the R programming
  language {[}4{]}. The Euler discretization brings approximation paths to
  stock prices and variance processes. If we set
  \(t_0 = 0 < t_1 < \dots < t_M = T\) as partitions of a time interval of
  \(M\) equal segments of lenght \(\delta t\), we have the following
  discretization for the stock price:
  \begin{align}
  S_{t+1} = S_t + rS_t + \sqrt{V_t} S_t Z_s
  \end{align}
  \noindent
  And for the variance process:
  \begin{align}
  V_{t+1} = V_t + k (\theta - V_t) + \sigma \sqrt{V_t} Z_v
  \end{align}
  \noindent
  \(Z_s\) being a standard normal random variable, i.e. \(N~(0,1)\), we
  set \(Z_t\) and \(Z_v\) as two independent standard normal random
  variables and \(Z_s\) and \(Z_v\) having correlation \(\rho\). This
  means we can write \(Z_s = \rho Z_v + \sqrt{1-\rho^2} Z_t\)
  
  \section{Karl Jaeckel}\label{karl-jaeckel}
  
  \section{Exact Algorithm}\label{exact-algorithm}
  
  \section{Teste}\label{teste}
  \begin{Shaded}
  \begin{Highlighting}[]
  \NormalTok{Call <-}\StringTok{ }\NormalTok{function(S,X,tau,r,q,sigma)\{}
    
    \CommentTok{# S = spot}
    \CommentTok{# X = strike}
    \CommentTok{# tau = time to maturity}
    \CommentTok{# r = riskfree rate}
    \CommentTok{# q = dividend yield}
    \CommentTok{# sigma = standard deviation}
    
    \NormalTok{d1 =}\StringTok{ }\NormalTok{(}\KeywordTok{log}\NormalTok{(S/X) +}\StringTok{ }\NormalTok{(r -}\StringTok{ }\NormalTok{q +}\StringTok{ }\NormalTok{sigma^}\DecValTok{2}\NormalTok{/}\DecValTok{2}\NormalTok{) *}\StringTok{ }\NormalTok{tau) /}\StringTok{ }\NormalTok{(}\KeywordTok{sqrt}\NormalTok{(sigma^}\DecValTok{2} \NormalTok{*}\StringTok{ }\NormalTok{tau))}
    \NormalTok{d2 =}\StringTok{ }\NormalTok{d1 -}\StringTok{ }\KeywordTok{sqrt}\NormalTok{(sigma^}\DecValTok{2} \NormalTok{*}\StringTok{ }\NormalTok{tau)}
    \NormalTok{c0 =}\StringTok{ }\NormalTok{S *}\StringTok{ }\KeywordTok{exp}\NormalTok{(-q *}\StringTok{ }\NormalTok{tau) *}\StringTok{ }\KeywordTok{pnorm}\NormalTok{(d1,}\DecValTok{0}\NormalTok{,}\DecValTok{1}\NormalTok{) -}
  \StringTok{       }\NormalTok{X *}\StringTok{ }\KeywordTok{exp}\NormalTok{(-r *}\StringTok{ }\NormalTok{tau) *}\StringTok{ }\KeywordTok{pnorm}\NormalTok{(d2,}\DecValTok{0}\NormalTok{,}\DecValTok{1}\NormalTok{)}
  
    \KeywordTok{return}\NormalTok{(c0)}
  \NormalTok{\}}
  \end{Highlighting}
  \end{Shaded}
  \clearpage
  \begin{Shaded}
  \begin{Highlighting}[]
  \KeywordTok{setwd}\NormalTok{(}\StringTok{"../rnmethods"}\NormalTok{)}
  \KeywordTok{source}\NormalTok{(}\StringTok{"../rnmethods/R/euler_heston.R"}\NormalTok{)}
  \end{Highlighting}
  \end{Shaded}
  \chapter{This chunk ensures that the thesisdown package
  is}\label{this-chunk-ensures-that-the-thesisdown-package-is}
  
  \chapter{Conclusion}\label{conclusion}
  
  \chapter{The First Appendix}\label{the-first-appendix}
  
  \chapter*{References}\label{references}
  \addcontentsline{toc}{chapter}{References}
  
  \hypertarget{refs}{}
  \hypertarget{ref-heston1993closed}{}
  {[}1{]} S.L. Heston, A closed-form solution for options with stochastic
  volatility with applications to bond and currency options, Review of
  Financial Studies. 6 (1993) 327--343.
  
  \hypertarget{ref-dunn2014estimating}{}
  {[}2{]} R. Dunn, P. Hauser, T. Seibold, H. Gong, Estimating option
  prices with heston's stochastic volatility model, (2014).
  
  \hypertarget{ref-broadie2006exact}{}
  {[}3{]} M. Broadie, Ö. Kaya, Exact simulation of stochastic volatility
  and other affine jump diffusion processes, Operations Research. 54
  (2006) 217--231.
  
  \hypertarget{ref-rlang}{}
  {[}4{]} R Core Team, R: A language and environment for statistical
  computing, R Foundation for Statistical Computing, Vienna, Austria,
  2017. \url{https://www.R-project.org/}.


  % Index?

\end{document}
